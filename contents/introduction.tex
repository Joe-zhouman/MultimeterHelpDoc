\chapter{前言}
\section{系统的开发背景和目的}
\par 界面结构是电子器件本身以及器件互连和封装中普遍存在的问题,这必
然会导致由于两接触界面产生的接触传热问题,因此有效的界面接触热阻控制
是电子产品性能和可靠性方面的重要问题。试验测试是获取接触热阻数据的重
要手段,最为常用的是稳态实验测量法。稳态实验测量法除了为在真空环境中
的测试对象提供八个小时以上持续稳定的高精度压力加载,还需要将采集原始
数据保存在本地,进行后续标定参数转化成实际温度和线性拟合计算热流量、
接触端面温度等计算工作。整个试验工作量大且效率低下,较易产生人为干预
造成的错误从而影响测量结果。
\par 针对上述问题,本公司提供该测试系统不仅能实现多种条件下的接触热
阻测试,而且可以实现人性化界面设置实验参数和数据自动处理求解试验结果。
本系统具有高精度、集成化、模块化和智能化的特点,可帮助测试人员轻松完
成测试,可完全代替手动、半自动等老旧设备,大大节约了测试人力成本,同
时也避免过多的人为干预造成的试验错误,提高测试自动化程度和测试精度。

\section{系统所能应用的领域和使用对象}
\par 本系统适用于机械制造、微电子、航空航天、化工、仪器仪表以及其他
相关工程领域的接触热阻测量。
\par 本系统的使用对象为研究院科研员、企业专业测试员和学校科研员等。

\section{系统的功能及特性简介}
\par 本系统用于接触热阻稳态试验测量,其中包括试件热导率测试、
固-固试件间接触热阻测试、热流计间热界面材料测试和固-固试件间热界面
材料测试等,具有以下特性:
\begin{itemize}
    \item 具有高精度、集成化、模块化和智能化等特点。
    \item 具有人性化界面设置实验参数,规范实验流程。
    \item 具有智能数据采集与处理功能,提高测试效率。
\end{itemize}
% \section{较上一版本的改进部分}
