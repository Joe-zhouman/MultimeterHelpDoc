\chapter{数据文件格式}
% 可以通过附录介绍用户必须了解或可以了解的各种输入数据
% 文件、输出结果文件、中间数据文件的格式、限制范围、适当的解释等。

% todo 配置文件说明
\section{配置文件}
\par 本软件采用\textcolor{red}{ini}文件作为配置文件,ini文件总是将章节(section),键(key),值(value)以特定格式组合在一起:
\begin{lstlisting}
    [section]
    key1 = value1
    key2 = value2
\end{lstlisting}
\par 下面的说明将按照section依次进行,对于key--value的说明按以下格式进行:
\begin{definition}{key}{}
    \begin{itemize}
        \item[说明] value的说明
        \item[允许值] value的允许值
        \item[备注] value的备注(可能没有该项) 
    \end{itemize}
\end{definition}

\begin{tips}{配置文件}{configFIle}
    配置文件直接影响着程序的运行,请勿随意修改,本章节的设立是希望软件的维护者对本软件有更清楚的认识,便于其进行维护工作。若要修改,请提前做好备份。
\end{tips}
\subsection{sys.ini}

sys.ini文件是目前本软件具有的唯一配置文件。
\subsubsection*{Serial}
串口设置
\begin{definition}{port}{}
    \begin{itemize}
        \item[说明] 串口类型
        \item[允许值] COM1至COM8
    \end{itemize}
\end{definition}

\begin{definition}{baudrate}{}
    \begin{itemize}
        \item[说明] 串口波特率
        \item[允许值] 4800, 9600, 14400, 19200, 38400, 57600, 115200
    \end{itemize}
\end{definition}

\begin{definition}{databits}{}
    \begin{itemize}
        \item[说明] 每个字节的标准数据位长度
        \item[允许值] 4,5,6,7,8
    \end{itemize}
\end{definition}

\begin{definition}{stopbites}{}
    \begin{itemize}
        \item[说明] 每个字节的标准停止位数
        \item[允许值] One, OnePointFive, Two
    \end{itemize}
\end{definition}

\begin{definition}{parity}{}
    \begin{itemize}
        \item[说明] 奇偶校验检查协议
        \item[允许值] Enen, Mark, None, Odd, Space
    \end{itemize}
\end{definition}

\subsubsection*{SYS}

系统设置

\begin{definition}{scanInterval}{}
    \begin{itemize}
        \item[说明] 单个通道扫描间隔
        \item[允许值] 单位为毫秒,大于scanIntervalLb,小于scanIntervalUb
    \end{itemize}
\end{definition}

\begin{definition}{scanIntervalLb}{}
    \begin{itemize}
        \item[说明] 单个通道扫描间隔设置下限
        \item[允许值] 单位为毫秒,小于scanIntervalUb
    \end{itemize}
\end{definition}

\begin{definition}{scanIntervalUb}{}
    \begin{itemize}
        \item[说明] 单个通道扫描间隔设置上限
        \item[允许值] 单位为毫秒,大于scanIntervalLb
    \end{itemize}
\end{definition}

\begin{definition}{saveInterval}{}
    \begin{itemize}
        \item[说明] 数据保存间隔
        \item[允许值] 单位为毫秒,大于saveIntervalLb,小于saveIntervalUb
    \end{itemize}
\end{definition}

\begin{definition}{saveIntervalLb}{}
    \begin{itemize}
        \item[说明] 数据保存间隔设置下限
        \item[允许值] 单位为毫秒,小于saveIntervalUb
    \end{itemize}
\end{definition}

\begin{definition}{saveIntervalUb}{}
    \begin{itemize}
        \item[说明] 数据保存间隔设置上限
        \item[允许值] 单位为毫秒,大于saveIntervalLb
    \end{itemize}
\end{definition}

\begin{definition}{autoSaveInterval}{}
    \begin{itemize}
        \item[说明] 温度数据收敛后多长时间自动停止测试
        \item[允许值] 单位为毫秒,不要太小
    \end{itemize}
\end{definition}

\begin{definition}{convergentLim}{}
    \begin{itemize}
        \item[说明] 温度数据收敛判断容差
        \item[允许值] 不要太小
    \end{itemize}
\end{definition}

\begin{definition}{alloweChannel}{}
    \begin{itemize}
        \item[说明] 允许使用的通道
        \item[允许值] 通道没之间以逗号连接,必须包含*,其他通道在101--140或201-240之间
    \end{itemize}
\end{definition}

\begin{definition}{tempLb}{}
    \begin{itemize}
        \item[说明] 测试容许温度下限
        \item[允许值] 单位$^{\circ}C$,0$^{\circ}C$左右
    \end{itemize}
\end{definition}

\begin{definition}{tempUb}{}
    \begin{itemize}
        \item[说明] 测试容许温度上限
        \item[允许值] 单位$^{\circ}C$,100$^{\circ}C$左右
    \end{itemize}
\end{definition}

\begin{definition}{method}{}
    \begin{itemize}
        \item[说明] 测试方法
        \item[允许值] KAPPA,ITC,ITM,ITMS
    \end{itemize}
\end{definition}

\subsubsection*{Card\#}
采集仪第\#张板卡的属性
\begin{definition}{enable}{}
    \begin{itemize}
        \item[说明] 板卡是否使用
        \item[允许值] 0,1
        \item[备注] 0--不使用,1--使用 
    \end{itemize}
\end{definition}

\subsubsection*{101--140,201--240}

采集频道及所连接的探头属性

\begin{definition}{type}{}
    \begin{itemize}
        \item[说明] 采集频道及所连接的探头类型
        \item[允许值] 0,1,2,3
        \item[备注] 0--未连接探头,1--电压探头,2--K型热电偶,3--四线电阻 
    \end{itemize}
\end{definition}

\begin{definition}{type}{}
    \begin{itemize}
        \item[说明] 采集频道及所连接的探头工作方式
        \item[允许值] 0,1,2,3
        \item[备注] 0--不工作,1--按电压探头工作,2--按K型热电偶工作,3--按四线电阻工作 
    \end{itemize}
\end{definition}

\begin{definition}{A\#}{}
    \begin{itemize}
        \item[说明] 采集频道及所连接的探头的第\#个标定参数
    \end{itemize}
\end{definition}

\subsubsection*{HeatMeter\#}
第\#个热流计的属性
\begin{definition}{kappa}{}
    \begin{itemize}
        \item[说明] 热流计热导率
        \item[允许值] 单位$W/(m\cdot K)$,大于0
    \end{itemize}
\end{definition}

\begin{definition}{area}{}
    \begin{itemize}
        \item[说明] 热流计横截面积
        \item[允许值] 单位$mm^2$,大于0
    \end{itemize}
\end{definition}
\subsubsection*{HeatMeter\#.\$}
第\#个热流计的从上至下第\$个测试点位的位置。

\begin{definition}{channel}{}
    \begin{itemize}
        \item[说明] 测试点位连接的频道
        \item[允许值] 101--140,201--240或*
    \end{itemize}
\end{definition}
\begin{definition}{position}{}
    \begin{itemize}
        \item[说明] 测试点位的位置坐标
        \item[允许值] 单位$mm$,大于0或*
    \end{itemize}
\end{definition}

\subsubsection*{Sample\#}
第\#个试件的属性

\begin{definition}{area}{}
    \begin{itemize}
        \item[说明] 试件横截面积
        \item[允许值] 单位$mm^2$,大于0
    \end{itemize}
\end{definition}

\subsubsection*{Sample\#.\$}
第\#个试件的从上至下第\$个测试点位的位置。

\begin{definition}{channel}{}
    \begin{itemize}
        \item[说明] 测试点位连接的频道
        \item[允许值] 101--140,201--240或*
    \end{itemize}
\end{definition}
\begin{definition}{position}{}
    \begin{itemize}
        \item[说明] 测试点位的位置坐标
        \item[允许值] 单位$mm$,大于0或*
    \end{itemize}
\end{definition}

\subsubsection*{ITM}
界面材料属性
\begin{definition}{thickness}{}
    \begin{itemize}
        \item[说明] 界面材料厚度
        \item[允许值] 单位$\mu m$,大于0
    \end{itemize}
\end{definition}
\begin{definition}{area}{}
    \begin{itemize}
        \item[说明] 界面材料横截面积
        \item[允许值] 单位$mm^2$,大于0
    \end{itemize}
\end{definition}

\subsubsection*{Pressure}
系统采用的压力
\begin{definition}{force}{}
    \begin{itemize}
        \item[说明] 界面施加的压力
        \item[允许值] 单位$kg$,大于0
    \end{itemize}
\end{definition}

\subsection{*.rst文件}
*.rst文件实际上是一种特殊的配置文件,除以上sys.ini内的内容外,其还包括计算结果所需的测试数据。

\subsubsection*{Data}
测试数据
\begin{definition}{\#}{}
    \begin{itemize}
        \item[说明] 第\#组数据
        \item[备注] 单位$^{\circ}C$,数据之间以逗号隔开。
    \end{itemize}
\end{definition}