\chapter{错误信息速查表\label{app:errorInfo}}
% 通常可以以表的形式按照一定的顺序,例如按出错提示信息编号顺序
% 或按出错提示信息的字母顺序,给出出错提示信息的编号、提示信息、相应的解释、出
% 错原因和解决办法。

% todo 错误信息汇总

\begin{table}[htbp]
    \centering
    \caption{ 输入参数时错误弹窗消息及可能的参数错误 \label{tab:errorInfoNormalUserA}}
    \begin{tabular}{@{}lcc@{}}
        \toprule
        弹窗消息                        & 错误参数   & 错误原因                                    \\ \midrule
        存在不合理的xx                  & Su,Sl,S,mm & 输入的不是一个数字                          \\
                                        & ITM,Force  & 输入的数字小于或等于0                       \\
        xx存在不可用频道                & chn        & 输入的不是一个正确的通道编号                \\
                                        &            & 通道不可用                                  \\
        xx的测温点太少                  & chn        & 组件xx已启用的点位小于3个                   \\
        存在除相同的频道                & chn        & 存在被重复使用的频道。                      \\
        xx未启用探测点的 & mm         & 当测试点位未启用时,对应的位置坐标未设置为* \\
        位置坐标应设为*& & \\
        xx已启用探测点的 & mm         & 已启用的测试点位对应的位置坐标被设置为*     \\
        位置坐标被设为* & & \\
        \bottomrule
    \end{tabular}
\end{table}
\begin{table}[htbp]
    \centering
    \caption{ 测试过程中可能遇到的错误及 \label{tab:errorInfo}}
    \begin{tabular}{@{}p{3cm}p{10cm}@{}}
        \toprule
        异常信息                        & 解决方案                                     \\ \midrule
        上下热流计热流相差过大& 检查实验装置电缸是否压紧、水冷机循环是否打开(按水冷机控制面板\textcolor{red}{'pump'}按钮可启动循环)、真空腔是否密封和真空泵是否打开等。\\
        数据采集异常,温度监视画面空白或不更新&尝试重启软件和数采仪。\\
        无法打开串口&检查数采仪是否开启、查看\textcolor{red}{电脑设备管理器}串口类型与本软件的设置是否一致。\\
        拟合数据过少,拟合结果不可靠&试件添加测温探头数量。\\ 
        测温探头位置信息和温度不对应&检查试件的测温探头编号设置是否正确\\ 
        程序由于未知错误崩溃&系软件开发商解决该问题\\
        \bottomrule
    \end{tabular}
\end{table}

